%
% File acl2016.tex
%
%% Based on the style files for ACL-2015, with some improvements
%%  taken from the NAACL-2016 style
%% Based on the style files for ACL-2014, which were, in turn,
%% Based on the style files for ACL-2013, which were, in turn,
%% Based on the style files for ACL-2012, which were, in turn,
%% based on the style files for ACL-2011, which were, in turn, 
%% based on the style files for ACL-2010, which were, in turn, 
%% based on the style files for ACL-IJCNLP-2009, which were, in turn,
%% based on the style files for EACL-2009 and IJCNLP-2008...

%% Based on the style files for EACL 2006 by 
%%e.agirre@ehu.es or Sergi.Balari@uab.es
%% and that of ACL 08 by Joakim Nivre and Noah Smith

\documentclass[11pt]{article}
\usepackage{acl2016}
\usepackage{times}
\usepackage{url}
\usepackage{latexsym}

\aclfinalcopy % Uncomment this line for the final submission
%\def\aclpaperid{***} %  Enter the acl Paper ID here

%\setlength\titlebox{5cm}
% You can expand the titlebox if you need extra space
% to show all the authors. Please do not make the titlebox
% smaller than 5cm (the original size); we will check this
% in the camera-ready version and ask you to change it back.

\newcommand\BibTeX{B{\sc ib}\TeX}

\title{Instructions for ACL-2016 Proceedings}

\author{Tom Byars, Cale Clark, Roman Fenlon, Charlie Lyttle, Katie McAskill, Jack Miller, Zein Said, \\
{\bf Abdullah Sayed, Laura Schauer, Jason Sweeney, Aron Szeles, Xander Wickham} \\
School of Mathematics and Computer Science, Heriot-Watt University, Edinburgh \\ {\tt {tjb10, cc164, rf104, cl157, klm12, jjm7, zs2008,}} \\ {\tt {as512, lms9, js418, as472, aw127}@hw.ac.uk}}

\date{February 2023}

\begin{document}
\maketitle
\begin{abstract}
  This should be a 6-8 page conference paper with appendices, if relevant. 
\end{abstract}


\section{Introduction}
\label{sec:introduction}

\textit{from coursework spec:} main research or technical question addressed


\section{Background}
\label{sec:background}

\textit{from coursework spec:} literature review / related work, including a critical analysis of the field, and commentary on applicability of the technologies and methods used in emerging technologies and application areas

\subsection{Socially Assistive Robots} % TODO: change "predicted to happen soon"
\label{subsec:socially_assistive_robots}
Socially assistive robots (SARs) are a crucial part of the future of many sectors, for example, in education or healthcare \cite{gunson_visually_aware_2022}. Especially the latter depends on technology advancements as it is facing numerous obstacles in the future, such as increasing spendings and a growing percentage of older people. A serious lack of healthcare workers is already occurring, with 10 million more healthworkers needed worldwide by 2030 \cite{cooper_ari_2020,Health_workforce_2023}. SARs have the capability to healthcare in various ways, such as encouraging older people to keep living independently or reducing emergency visits \cite{cooper_ari_2020}.

\subsection{SPRING-ARI}
\label{subsec:spring_ari}
Previously, development has been conducted on a SAR robot deployed in an eldercare hospital reception area \cite{addlesee_comprehensive_2020}. \\
\textit{Explain what has been done so far with SPRING-ARI.}

\subsection{Requirements of a SAR in healthcare}
\label{subsec:SAR_requirements}
The difficulty of designing such a SAR lies in creating robots that can effectively converse with humans and adhere to social norms \cite{moujahid_multi_party_2022}. More expressive robots seem more intelligent, conscious and polite to the people that interact with it \cite{moujahid_multi_party_2022}. To achieve this positive perception, multiple parts need to be combined into one conversational system, such as the ability to carry out visually grounded as well as task-based dialogues, to perceive and discuss its environment and to chit-chat \cite{gunson_visually_aware_2022}. In particular, multi-party engagement and fluent-turn taking in multi-party environments is becoming more important as dialogue systems improve beyond simpler task-based systems \cite{skantze_turn_taking_2021}. However, state-of-the-art models for training such robots are still too simplistic \cite{skantze_turn_taking_2021}. There is a need to 

\subsection{State-of-the-art Tools}
\label{subsec:tools}
\textit{Explain what tools we'll be using}


\section{Design and Implementation}
\label{sec:implementation}

\textit{from coursework spec:} design and implementation of the system: components and architecture


\section{Evaluation}
\label{sec:evaluation}

\textit{from coursework spec:} evaluation of the system and presentation of the results


\section{Future Work}
\label{sec:future_work}

\textit{from coursework spec:} suggestions for future work


\section*{Acknowledgments}
\label{sec:acknowledgments}

The acknowledgments should go immediately before the references.  Do
not number the acknowledgments section. Do not include this section
when submitting your paper for review.

% include your own bib file like this:
%\bibliographystyle{acl}
%\bibliography{acl2016}
\bibliography{references}
\bibliographystyle{acl2016}

\appendix

\section{Supplemental Material, Appendix}
\label{sec:supplemental}


\end{document}
