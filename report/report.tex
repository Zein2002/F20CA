%
% File acl2016.tex
%
%% Based on the style files for ACL-2015, with some improvements
%%  taken from the NAACL-2016 style
%% Based on the style files for ACL-2014, which were, in turn,
%% Based on the style files for ACL-2013, which were, in turn,
%% Based on the style files for ACL-2012, which were, in turn,
%% based on the style files for ACL-2011, which were, in turn, 
%% based on the style files for ACL-2010, which were, in turn, 
%% based on the style files for ACL-IJCNLP-2009, which were, in turn,
%% based on the style files for EACL-2009 and IJCNLP-2008...

%% Based on the style files for EACL 2006 by 
%%e.agirre@ehu.es or Sergi.Balari@uab.es
%% and that of ACL 08 by Joakim Nivre and Noah Smith

\documentclass[hidelinks, 11pt]{article}
\usepackage{acl2016}
\usepackage{times}
\usepackage{url}
\usepackage{latexsym}

% For referencing sections across the article
\usepackage{nameref}
\usepackage[]{hyperref}

\aclfinalcopy % Uncomment this line for the final submission
%\def\aclpaperid{***} %  Enter the acl Paper ID here

%\setlength\titlebox{5cm}
% You can expand the titlebox if you need extra space
% to show all the authors. Please do not make the titlebox
% smaller than 5cm (the original size); we will check this
% in the camera-ready version and ask you to change it back.

\newcommand\BibTeX{B{\sc ib}\TeX}

\title{Instructions for ACL-2016 Proceedings}

\author{Tom Byars, Cale Clark, Roman Fenlon, Charlie Lyttle, Katie McAskill, Jack Miller, Zein Said, \\
{\bf Abdullah Sayed, Laura Schauer, Jason Sweeney, Aron Szeles, Xander Wickham} \\
School of Mathematics and Computer Science, Heriot-Watt University, Edinburgh \\ {\tt {tjb10, cc164, rf104, cl157, klm12, jjm7, zs2008,}} \\ {\tt {as512, lms9, js418, as472, aw127}@hw.ac.uk}}

\date{February 2023}

\begin{document}
\maketitle
\begin{abstract}
  This should be a 6-8 page conference paper with appendices, if relevant.
  Good reports from last year: 1 and 7
\end{abstract}


\section{Introduction}
\label{sec:introduction}

\textit{from coursework spec:} main research or technical question addressed \\
Socially assistive robots (SARs) are a crucial part of the future of many sectors, for example, in education or healthcare \cite{gunson_visually_aware_2022}. Especially the latter depends on technology advancements as it is facing numerous obstacles in the future, such as increasing spendings and a growing percentage of older people. A serious lack of healthcare workers is already occurring, with 10 million more healthworkers needed worldwide by 2030 \cite{cooper_ari_2020,Health_workforce_2023}. SARs can pose a solution to the problem, as they are able to support healthcare in various ways, such as encouraging older people to keep living independently for longer or reducing caregiver burden \cite{cooper_ari_2020}.

These scenarios require SARs to be able to handle multi-party interactions as it is likely that more than one person will interact with the system. Compared to handling dyadic interactions, handling multi-party conversations includes more complex challenges, such as Speaker Recognition, Addressee Recognition, Response Selection (summarised in “who says what to whom“) and turn-taking \cite{Group_1_unpublished_paper,Johansson_Skantze_2015}.

\textit{Include here what exactly we examined about turn-taking}

In this work, we propose a model trained on multi-party human-human conversation data. We collected the data from recordings of special “Who wants to be millionaire?“ episodes where two candidates collaborated to answer the host's questions.

\textit{Include results here.}


\section{Background}
\label{sec:background}

\textit{from coursework spec:} literature review / related work, including a critical analysis of the field, and commentary on applicability of the technologies and methods used in emerging technologies and application areas

\subsection{Socially Assistive Robots}
\label{subsec:socially_assistive_robots}
For healthcare, as well as for any other sector, the difficulty of successfully designing SARs lies in creating robots that can effectively converse with humans and adhere to social norms \cite{moujahid_multi_party_2022}. The more expressive a robot is, the more it will be perceived as intelligent, conscious and polite \cite{moujahid_multi_party_2022}. To achieve such a positive perception, multiple parts need to be combined into one conversational system, such as the ability to carry out visually grounded as well as task-based dialogues, to perceive and discuss its environment and to chit-chat \cite{gunson_visually_aware_2022}.

The SPRING project conducts research on a SAR robot that is deployed in an eldercare hospital reception area \cite{addlesee_comprehensive_2020}. The conversational system is deployed on humanoid ARI robot produced by Pal Robotics \cite{palrobot}. ARIs capabilities can be extended with custom AI algorithms, in the case of SPRING-ARI a visual perception system, a dialogue system, and a social interaction planner \cite{addlesee_comprehensive_2020}. While the SPRING-ARI system successfully demonstrates that task-based, social and visually grounded dialogue can be combined with physical actions, it still lacks the ability to handle conversations with more than one person simultaneously \cite{addlesee_comprehensive_2020}.

\subsection{Multi-party Human Robot Interaction}
\label{subsec:multi_party}
As stated above, the endeavour to create conversational systems becomes considerably more difficult when dealing with multi-party interactions \cite{Group_1_unpublished_paper}. Especially turn-taking poses a central problem. It is defined as follows:

\begin{quote}
  The rules of turn-taking organize the conversation into turns, during which one of the participants has the right to speak while the others agree to listen \cite{Żarkowski_2019}
\end{quote}

In dyadic conversations, there are only two roles a participant can take: speaker or listener, hence it is clear when and to whom the turn is yielded. In multi-party conversations, participants can take multiple roles, therefore turn-taking needs to be coordinated \cite{Johansson_Skantze_2015}. Humans signal their intents mostly through gaze, but also through pauses, prosody, and body positioning \cite{Żarkowski_2019}. To copy this behaviour, earlier models for conversational systems relied on silence time-outs to coordinate turn-taking, however, this approach is found to be too simplistic \cite{skantze_turn_taking_2021}. Instead, mimicking human turn-taking behaviour better by using a combination of verbal and non-verbal cues leads to robots that are better perceived \cite{moujahid_multi_party_2022}.

\textit{State exactly the gap that we will fill - whatever that will be}


\section{Data Collection}
\label{sec:data_collection}

\begin{itemize}
  \item (Laura) Talk about multi-party data collection
  \item (Aron and Katie) How we collected our data, describe the intents we used to label the data
  \item (Aron and Katie) Cohen's Kappa for our data collection method: probably need to annotate a couple of transcripts twice for reporting on this
\end{itemize}

\section{Design and Implementation}
\label{sec:implementation}

\begin{itemize}
  \item \textit{from coursework spec:} design and implementation of the system: components and architecture
  \item Jack's System Graph
\end{itemize}

\subsection{Natural Language Understanding}
\label{subsec:nlu}

\begin{itemize}
  \item Very short escription of RASA
  \item Refer to the intents described in \ref{sec:data_collection} \nameref{sec:data_collection}
  \item Different versions of the model: show difference in F-Score, Confusion Matrix, ... (Laura: look up in literature what is used, what should be used)
\end{itemize}

\subsection{Dialogue Management}
\label{subsec:dialogue_management}

\begin{itemize}
  \item Clearly explain 2 parts (State-Machine and NN)
  \item State-Machine: high-level control, handles things we have no data for, eg. pauses
  \item NN: Report on differences between RNN and LSTM and why the choice for the LSTM has been made
  \item (optional): Comparison to RASA rule-policy
\end{itemize}

\subsection{Natural Language Generation}
\label{subsec:nlg}


\section{Evaluation}
\label{sec:evaluation}

\textit{from coursework spec:} evaluation of the system and presentation of the results

\subsection{Methodology}
\label{subsec:methodology}

\subsection{Experiment Layout}
\label{subsec:experiment_layout}

\subsection{Results}
\label{subsec:results}

\section{Conclusion}
\label{sec:conclusion}

\subsection{Ethical Reflection}
\label{subsec:ethics}

\section{Future Work}
\label{sec:future_work}

\textit{from coursework spec:} suggestions for future work


\section*{Acknowledgments}
\label{sec:acknowledgments}

The acknowledgments should go immediately before the references.  Do
not number the acknowledgments section. Do not include this section
when submitting your paper for review.

% include your own bib file like this:
%\bibliographystyle{acl}
%\bibliography{acl2016}
\bibliography{references}
\bibliographystyle{acl2016}

\appendix

\section{Supplemental Material, Appendix}
\label{sec:supplemental}


\end{document}
